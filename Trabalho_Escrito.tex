\documentclass[a4paper, 12pt]{article}

\usepackage[portuges]{babel}
\usepackage[utf8]{inputenc}
\usepackage{amsmath}
\usepackage{indentfirst}
\usepackage{blindtext}
\usepackage{graphicx}
\usepackage[hidelinks]{hyperref}
\usepackage{gensymb}
\usepackage[usenames, dvipsnames]{xcolor}

\author{Igor Abreu da Silva}

\title{Trabalho Final de Sistemas Lineares I}

\begin{document}

	\begin{titlepage}
		\begin{center}
			\huge{Universidade Federal do Rio de Janeiro}
			\vspace{95pt}

			\large{Trabalho Final de Sistemas Lineares I}
			\vspace{160pt}
		\end{center}

		\begin{flushleft}
			\begin{tabbing}
				Alunos\qquad\qquad\= Igor Abreu da Silva\\
				DRE\> 112053874 \\
				Curso\> Engenharia Eletrônica \\
				Turma\> 2016/2 \\
				Professor\> Natanael Nunes de Moura Junior \\

			\end{tabbing}

		\end{flushleft}

		\begin{center}
			\vspace{\fill}
			Rio de Janeiro, 09 de Dezembro de 2016
		\end{center}
	\end{titlepage}


	\newpage
	\tableofcontents
	\listoffigures
	\thispagestyle{empty}

	\newpage
	\pagenumbering{arabic}
	\section{Conhecimentos}\label{conhecimentos}
		\subsection{Leis de kirchhoff}\label{kirchhoff}
		\subsubsection{Lei das Correntes}
			A soma das correntes em um nó e igual a zero.
		\subsubsection{Lei das Tensões}
			A soma das tensões em uma malha fechada é nula.
		\subsection{Propriedades de Laplace}
			\subsubsection{Propriedade da diferenciação}\label{derivada}
			\[
			x(t) \Rightarrow X(S) \rightarrow \int_{0}^{+\infty} \frac{dx}{dt}e^{-st}dt \Rightarrow s\int_{0}^{\infty} x(t)e^{-st}dt - x(0^{-}) = sX(s) - x(0^{-})
			\]
		\subsection{Função de Transferência}
		É a relação da saida sobre a entradas $H(S) = \frac{Y(S)}{X(S)}$
		\subsection{Pólos e Zeros}\label{zero}
		Pólos são valores de \textbf{s} onde temos uma indefinição na função de transferência, ou seja quando o denominador é zero.


		Zeros são todos os valores de \textbf{s} que zera a função de transferência, em outras palavras fazem com que o numerador seja igual a zero.
		\subsection{Diagrama de Bode}\label{bode}
		\[
		H(S) = \frac{Y(s)}{X(S)} \Rightarrow \frac{\textcolor{blue}{k}(\textcolor{red}{s+a_{1}})}{\textcolor{Brown}{s}(\textcolor{Violet}{s+b_{1}})(\textcolor{Orange}{s^{2}+b_{2}s+b_{3}})}
		\]
		Organizando esta equação, temos:
		\[
		 \frac{\textcolor{blue}{k}(\textcolor{red}{a_{1}(\frac{s}{a_{1}}+1)})}{\textcolor{Brown}{s}\textcolor{Violet}{b_{1}}\textcolor{Orange}{b_{3}}(\textcolor{Violet}{\frac{s}{b_{1}}+1})(\textcolor{Orange}{\frac{s^{2}}{b_{3}}+\frac{b_{2}s}{b_{3}}+1})}
		\]
		Com isso, teremos:
		\[
		|H(jw)| = 		 \frac{\textcolor{blue}{k}(\textcolor{red}{a_{1}(\frac{jw}{a_{1}}+1)})}{\textcolor{Brown}{jw}\textcolor{Violet}{b_{1}}\textcolor{Orange}{b_{3}}(\textcolor{Violet}{\frac{jw}{b_{1}}+1})(\textcolor{Orange}{\frac{(jw)^{2}}{b_{3}}+\frac{b_{2}jw}{b_{3}}+1})}
		\]
		Por questões praticas, com o objetivo de facilitar as contas, adaptaremos essa função para utilizar como a representação magnitude de decibel, logo teremos:\\

		\[20log(|H(jw)|) = 20log(|\frac{ka_{1}}{b_{1}b{3}}|) + 20log(|\frac{jw}{a_{1}}+1|) - 20log(|jw|) -20log(|\frac{jw}{b_{1}}+1|)
		\]
		\[
		- 20log(|\frac{(jw)^{2} + b_{2}jw}{b_{3}}+1|)
		\]
		Como fase, teremos:

		\[
		\angle H(jw) = \angle (\frac{jw}{a_{1}}+1) - \angle jw - \angle (\frac{jw}{b_{1}}+1) - \angle (\frac{(jw)^{2} + b_{2}jw}{b_{3}}+1)
		\]

		\textbf{Para o termo constante: $\frac{ka_{1}}{b_{1}b{3}}$}

		Modulo começa em $20log(\frac{ka_{1}}{b_{1}b{3}})$

		Fase será $\pi, \frac{ka_{1}}{b_{1}b{3}} > 0$ ou $0, \frac{ka_{1}}{b_{1}b{3}} \leq 0$\\

		\textbf{Termo: $j\omega$}

		Modulo -20db/dec (Pólo) ou +20db/dec (zero)

		Fase $-arctg(\frac{w}{0})$, ou seja, $-90^{o}$ (pólo) $+90^{o}$ (zero)\\

		\textbf{Primeira Ordem: $\frac{jw}{b_{1}}+1$}

		Modulo, começando na frequência de corte $b_{1}$, -20db/dec (Pólo) ou +20db/dec (zero)

		Fase $0_{o}$ em $\frac{b_{1}}{10}$, $-45^{o}$ (pólo) $+45^{o}$ (zero) em $b_{1}$ e por fim $-90^{o}$ (pólo) $+90^{o}$ (zero) em $100b_{1}$\\

		\textbf{Segunda Ordem: $\frac{(jw)^{2} + b_{2}jw}{b_{3}}+1$}

		Modulo, começando na frequência de corte $b_{3}$, -40db/dec (Pólo) ou +40db/dec (zero)

		Fase $0_{o}$ em $\frac{b_{3}}{10}$, $-90^{o}$ (pólo) $+90^{o}$ (zero) em $b_{3}$ e por fim $-180^{o}$ (pólo) $+180^{o}$ (zero) em $100b_{3}$\\
		\subsection{Resposta à sinas}
		Exemplo de como encontrar à resposta da função de transferência para um determinado sinal:
		
		Tendo $H(S) = \frac{S+2}{S^{2} + 5S + 4}$, iremos encontrar a resposta para $x(t) = 5cos(2t + 30\degree)$
		\[
		H(j\omega) = \frac{j\omega + 2}{-\omega^{2} + 5j\omega + 4}
		\]
		sabemos que:
		\[
		\arrowvert H(j\omega) \arrowvert = \frac{\sqrt{(\rm I\!Re_{1})^{2} + (\rm Im_{1})^{2}}}{\sqrt{(\rm I\!Re_{2})^{2} + (\rm Im_{2})^{2}}} \leftrightarrow \angle = \arctan\left(\frac{\rm Im_{1}}{\rm I\!Re_{1}}\right) - \arctan\left(\frac{\rm Im_{2}}{\rm I\!Re_{2}}\right)
		\]
		e que:
		\[
		y(t) = \arrowvert H(j\omega) \arrowvert \cos(\omega t + \phi + \angle H(j\omega))
		\]
		logo, substituindo os valores em:
		\[
		\arrowvert H(j\omega) \arrowvert = \frac{\sqrt{\omega^{2} + 4}}{\sqrt{(5j\omega)^{2} + (4 - \omega^{2})}} \leftrightarrow \angle H(j\omega) = \arctan\left(\frac{\omega}{2}\right) - \arctan\left(\frac{5\omega}{4 - \omega^{2}}\right)
		\]
		temos:
		\[
		y(t) = \sqrt{2}\cos(2t - 15\degree)
		\]		
		\subsection{Série Trigonométrica de Fourier}
		A série de fourier serve para representar sinais periódicos em função de senos e cossenos.
		
		com isso, temos:
		\[\frac{a_{0}}{2} + \sum_{n=1}^{+\infty}(a_{n}cos(\frac{2n\pi x}{T}) + b_{n}sen(\frac{2n\pi x}{T}))\]
		\[a_{n} = \frac{2}{T}\int_{T_{0}}^{T_{f}}f(x)cos(\frac{2n\pi x}{T})dx\]		
		\[b_{n} = \frac{2}{T}\int_{T_{0}}^{T_{f}}f(x)sen(\frac{2n\pi x}{T})dx\]				
	\newpage
	\section{Quest\~{a}o 1}\label{q1}
		Nesta sessão serão resolvidas todas as partes necessárias para encontrar as funções/utilidades de cada um dos circuitos, bem como a analise de resposta a determinados sinais e todos os itens solicitados na \textbf{Questão 1} do trabalho final de sistemas lineares.

		O primeiro passo é modelar cada circuito, essa modelagem utilizara as leis de kirchhoff explicadas em \ref{kirchhoff}. Apos a modelagem, encontraremos as E.D.O's, aplicaremos Laplace utilizando a propriedade da Derivação (\ref{derivada}) e com isso encontraremos a função de transferência H(S). Em posse da função de transferência em função dos componentes do circuito, escolheremos o valor comercial de cada componente de acordo com o nosso objetivo, encontraremos os polos e zeros e o diagrama de bode conforme explicado em (\ref{zero}) e (\ref{bode}).

		\subsection{Circuito 1}

			\subsubsection{Determinar a função do circuto}
			\begin{figure}[!ht]
				\centering
				\includegraphics{img/circuito1.png}
				\caption{Circuito 1}
			\end{figure}
			Podemos modelar o circuito 1 em relação ao nó após R1. Teríamos a seguinte equação: \\
			\[
				\frac{V_{in} - V_{out}}{R1} - \frac{V_{out}}{R2} - \frac{C\partial V_{out}}{\partial t} - \frac{1}{L} \int V_{out}{\partial t} = 0
			\] 	\\
			Para encontrarmos a E.D.O do circuito, vamos derivar toda esta expressão e separar $V_{out}$ e $V_{in}$, encontrando a seguinte relação:
			\[
				\frac{\partial V_{in}}{\partial t} \left(\frac{1}{R_{1}}\right) = \frac{C \partial^{2} V_{out}}{\partial t^{2}} + \frac{\partial V_{out}}{\partial t} \left(\frac{1}{R_{1}} + \frac{1}{R_{2}} \right) + \frac{V_{out}}{L}
			\] 	\\
			Em posse da E.D.O, utilizaremos Laplace para encontrar a função de Transferência do Circuito.
			\[
				X(S) \left(\frac{1}{R_{1}}\right) = Y(S)\left(S^{2}C + S \left(\frac{1}{R_{1}} + \frac{1}{R_{2}}\right) + \frac{1}{L} \right) \Rightarrow
			\] 	\\
			\[
				H(S) = \frac{Y(S)}{X(S)} = \frac{SR_{2}L}{S^{2}\left(R_{1}R_{2}LC\right) + S\left(R_{1}L + R_{2}L\right) + R_{1}R_{2}}
			\] 	\\

			Afim de facilitar os cálculos, tomaremos os seguintes valores para cada elemento do circuito:
			 \begin{itemize}
				\item $R_{1} = 56 \Omega;$
				\item $R_{2} = 100 \Omega;$
				\item $C = 2.2\mu F;$
				\item $L = 23.2mH;$
			 \end{itemize}

			Apos aplicar os valores comercias em H(S), temos:
			\[
			H(S) = \frac{2.32S}{0.000285824S^{2} + 3.6192S + 5600}
			\] 	\\

			Utilizando essa função no MatLab para encontrar os polos (quando se zera o denominador), zeros (quando se zera o numerador) e o diagrama de Bode, obtemos o seguintes gráficos:
			\newpage
			\begin{figure}[!ht]
				\centering
				\includegraphics[scale=0.7]{img/1e_circ1.png}
				\caption{Circuito 1 - Polos e Zeros}
			\end{figure}

			\begin{figure}[!ht]
				\centering
				\includegraphics[scale=0.3]{img/1f_circ1.png}
				\caption{Circuito 1 - Diagrama de Bode}
			\end{figure}

			Analisando-se este circuito, pode-se afirmar que o mesmo é um filtro passa faixa operando na largura de banda de aproximadamente 12610 rad/sec em um intervalo [1390, 14000] rad/sec.
			\newpage
			\subsubsection{Resposta ao degrau unitário}
			\begin{figure}[!ht]
				\centering
				\includegraphics[scale=0.68]{img/1g_circ1.png}
				\caption{Circuito 1 - Resposta ao degrau unitário}
			\end{figure}
			\subsubsection{Resposta a rampa unitário}
			\begin{figure}[!ht]
				\centering
				\includegraphics[scale=0.68]{img/1h_circ1.png}
				\caption{Circuito 1 - Resposta a rampa unitária}
			\end{figure}

			\subsubsection{Resposta a onda quadrada}
			\begin{figure}[!ht]
				\centering
				\includegraphics[scale=0.71]{img/1i_circ1.png}
				\caption{Circuito 1 - Resposta a onda quadrada com $\omega = \frac{1}{8}\pi$}
			\end{figure}
			\begin{figure}[!ht]
				\centering
				\includegraphics[scale=0.71]{img/1j_circ1.png}
				\caption{Circuito 1 - Resposta ao primeiro harmônico da série de Fourier de um onda quadrada com $\omega = \frac{1}{8}\pi$}
			\end{figure}
			\begin{figure}[!ht]
				\centering
				\includegraphics[scale=0.71]{img/1k_circ1.png}
				\caption{Circuito 1 - Resposta ao terceiro harmônico da série de Fourier de um onda quadrada com $\omega = \frac{1}{8}\pi$}
			\end{figure}
			\begin{figure}[!ht]
				\centering
				\includegraphics[scale=0.71]{img/1l_circ1.png}
				\caption{Circuito 1 - Resposta ao quinto harmônico da série de Fourier de um onda quadrada com $\omega = \frac{1}{8}\pi$}
			\end{figure}
			\begin{figure}[!ht]
				\centering
				\includegraphics[scale=0.71]{img/1m_circ1.png}
				\caption{Circuito 1 - Resposta ao sétimo harmônico da série de Fourier de um onda quadrada com $\omega = \frac{1}{8}\pi$}
			\end{figure}
		\newpage
		\subsection{Circuito 2}
			\subsubsection{Determinar a função do circuto}
			\begin{figure}[!hb]
				\centering
				\includegraphics{img/circuito2.png}
				\caption{Circuito 2}
			\end{figure}

			Para modelarmos utilizaremos as seguintes equações:
			\[
				 I_{1} = I_{in} - I_{out}
			\] 	\\
			\[
				 R_{2}I_{out} + \frac{L \partial I_{out}}{\partial t} - R_{1}I_{1} + \frac{1}{C}\int I_{out}\partial t = 0
			\] 	\\

			Substituindo $I_{1}$ para colocarmos a equação em função de $I_{in}$ e $I_{out}$ e derivando-a para removermos a Integral, temos a E.D.O:

			\[
				\frac{\partial I_{in}}{\partial t}\left(R_{1}\right) = \frac{\partial^{2}I_{out}}{\partial t^{2}}\left(L\right) + \frac{\partial I_{out}}{\partial t}\left(R_{1} + R_{2}\right) + \frac{I_{out}}{C}
			\] 	\\

			Transformando essa E.D.O em Laplace, obtemos:

			\[
				X(S)\left(SR_{1}\right) = Y(S)\left(S^{2} + S\left(R_{1} +  R_{2}\right) + \frac{1}{C}\right) \Rightarrow
			\] 	\\
			\[
			H(S) = \frac{Y(S)}{X(S)} = \frac{S\left(R_{1}C\right)}{S^{2}\left(LC\right) + S\left(R_{1}C + R_{2}C\right) + 1}
			\] 	\\

			Escolhendo os seguintes valores para cada elemento do circuito:
			\begin{itemize}
				\item $R_{1} = 1M \Omega;$
				\item $R_{2} = 100 \Omega;$
				\item $C = 2.2\mu F;$
				\item $L = 23.2mH;$				
			\end{itemize}

			Encontramos a seguinte função de transferência:
			\[
				H(S) = \frac{2.2S}{0.000000051S^{2} + 2.20022S + 1}
			\] 	\\
			A partir dessa função obtemos os seguintes polos, zeros e diagrama de Bode:
			\begin{figure}[!ht]
				\centering
				\includegraphics[scale=0.8]{img/1e_circ2.png}
				\caption{Circuito 2 - Polos e Zeros}
			\end{figure}

			\begin{figure}[!ht]
				\centering
				\includegraphics[scale=0.7]{img/1f_circ2.png}
				\caption{Circuito 2 - Diagrama de Bode}
			\end{figure}
			\newpage
			Assim como o circuito da figura 1, temos também um filtro passa faixa que opera nas faixas entre 0.427 rad/seg até $48.1\times10^{7}$ rad/seg

			\subsubsection{Resposta ao degrau unitário}
			\begin{figure}[!ht]
				\centering
				\includegraphics[scale=0.68]{img/1g_circ2.png}
				\caption{Circuito 2 - Resposta ao degrau unitário}
			\end{figure}
			\subsubsection{Resposta a rampa unitário}
			\begin{figure}[!ht]
				\centering
				\includegraphics[scale=0.68]{img/1h_circ2.png}
				\caption{Circuito 2 - Resposta a rampa unitária}
			\end{figure}

			\subsubsection{Resposta a onda quadrada}
			\begin{figure}[!ht]
				\centering
				\includegraphics[scale=0.71]{img/1i_circ2.png}
				\caption{Circuito 2 - Resposta a onda quadrada com $\omega = \frac{1}{8}\pi$}
			\end{figure}
			\begin{figure}[!ht]
				\centering
				\includegraphics[scale=0.71]{img/1j_circ2.png}
				\caption{Circuito 2 - Resposta ao primeiro harmônico da série de Fourier de um onda quadrada com $\omega = \frac{1}{8}\pi$}
			\end{figure}
			\begin{figure}[!ht]
				\centering
				\includegraphics[scale=0.71]{img/1k_circ2.png}
				\caption{Circuito 2 - Resposta ao terceiro harmônico da série de Fourier de um onda quadrada com $\omega = \frac{1}{8}\pi$}
			\end{figure}
			\begin{figure}[!ht]
				\centering
				\includegraphics[scale=0.71]{img/1l_circ2.png}
				\caption{Circuito 2 - Resposta ao quinto harmônico da série de Fourier de um onda quadrada com $\omega = \frac{1}{8}\pi$}
			\end{figure}
		    \clearpage
			\begin{figure}[!ht]
				\centering
				\includegraphics[scale=0.71]{img/1m_circ2.png}
				\caption{Circuito 2 - Resposta ao sétimo harmônico da série de Fourier de um onda quadrada com $\omega = \frac{1}{8}\pi$}
			\end{figure}
		\subsection{Circuito 3}
			\subsubsection{Determinar a função do circuto}
			\begin{figure}[!ht]
				\centering
				\includegraphics{img/circuito3.png}
				\caption{Circuito 3}
			\end{figure}

			Este circuito, também conhecido como topologia de Sallen-Key, sabendo que o AmpOp possui impedância infinita em sua entrada, que $V^{-} = V^{+}$, que $V^{-} = V_{out}$ e chamando $V_{a}$ da tensão que passa por $C_{1}$, obtemos:

			\[
			V_{a} = V_{out} + R_{2}C_{2}\frac{\partial V_{out}}{\partial t}
			\] 	\\

			Utilizando a lei dos nós entre $R_{1}$ e $R_{2}$ e já substituindo $V_{a}$ por $V_{out}$ temos:

			\[
			\frac{V_{in}}{R_{1}} = R_{2}C_{1}C_{2}\frac{\partial^{2} V_{out}}{\partial t^{2}} + \left(C_{2} + \frac{R_{2}C_{2}}{R_{1}}\right)\frac{\partial V_{out}}{\partial t} +  \frac{V_{out}}{R_{1}}
			\] 	\\

			Com esta E.D.O, podemos encontrar a seguinte função de transferência utilizando o mesmo método empregado nos circuitos anteriores, com isso temos:

			\[
			H(S) = \frac{1}{S^{2}\left(R_{1}R_{2}C_{1}C_{2}\right) + S\left(R_{1}C_{2} + R_{2}C_{2}\right) + 1}
			\] 	\\

			Utilizando os valores para cada elemento do circuito:
			\begin{itemize}
				\item $R_{1} = 10k\Omega;$
				\item $R_{2} = 1k\Omega;$
				\item $C_{1} = 50nF;$
				\item $C_{2} = 50nF;$
			\end{itemize}

			Encontramos a seguinte função de transferência:
			\[
			H(S) = \frac{1}{0.000000025S^{2} + 0.00055S + 1}
			\] 	\\

			Que nos gera os seguintes polos, zeros e diagrama de Bode:
			\begin{figure}[!ht]
				\centering
				\includegraphics[scale=0.7]{img/1e_circ3.png}
				\caption{Circuito 3 - Polos e Zeros}
			\end{figure}
			\newpage
			\begin{figure}[!ht]
				\centering
				\includegraphics[scale=0.7]{img/1f_circ3.png}
				\caption{Circuito 3 - Diagrama de Bode}
			\end{figure}
			Pela a analise do diagrama de Bode, pode-se afirmar que esse circuito é um filtro passa baixa com frequência de corte igual a 2000 rad/s.
			\subsubsection{Resposta ao degrau unitário}
			\begin{figure}[!ht]
				\centering
				\includegraphics[scale=0.68]{img/1g_circ3.png}
				\caption{Circuito 3 - Resposta ao degrau unitário}
			\end{figure}
			\subsubsection{Resposta a rampa unitário}
			\begin{figure}[!ht]
				\centering
				\includegraphics[scale=0.68]{img/1h_circ3.png}
				\caption{Circuito 3 - Resposta a rampa unitária}
			\end{figure}

			\subsubsection{Resposta a onda quadrada}
			\begin{figure}[!ht]
				\centering
				\includegraphics[scale=0.71]{img/1i_circ3.png}
				\caption{Circuito 3 - Resposta a onda quadrada com $\omega = \frac{1}{8}\pi$}
			\end{figure}
			\begin{figure}[!ht]
				\centering
				\includegraphics[scale=0.71]{img/1j_circ3.png}
				\caption{Circuito 3 - Resposta ao primeiro harmônico da série de Fourier de um onda quadrada com $\omega = \frac{1}{8}\pi$}
			\end{figure}
			\begin{figure}[!ht]
				\centering
				\includegraphics[scale=0.71]{img/1k_circ3.png}
				\caption{Circuito 3 - Resposta ao terceiro harmônico da série de Fourier de um onda quadrada com $\omega = \frac{1}{8}\pi$}
			\end{figure}
			\begin{figure}[!ht]
				\centering
				\includegraphics[scale=0.68]{img/1l_circ3.png}
				\caption{Circuito 3 - Resposta ao quinto harmônico da série de Fourier de um onda quadrada com $\omega = \frac{1}{8}\pi$}
			\end{figure}
			\begin{figure}[!ht]
				\centering
				\includegraphics[scale=0.68]{img/1m_circ3.png}
				\caption{Circuito 3 - Resposta ao sétimo harmônico da série de Fourier de um onda quadrada com $\omega = \frac{1}{8}\pi$}
			\end{figure}
		\subsection{Circuito 4}
			\subsubsection{Determinar a função do circuto}
			\begin{figure}[!ht]
				\centering
				\includegraphics{img/circuito4.png}
				\caption{Circuito 4}
			\end{figure}

			Esse circuito, conhecido como buffer, é utilizado como um isolador. Como $V_{in}$ é igual a $V_{out}$, sua função de transferência  H(S) = 1. Não existem polos nem zeros para esse circuito e seu diagrama de Bode permanece em 0.
			\begin{figure}[!ht]
				\centering
				\includegraphics[scale=0.7]{img/1f_circ4.png}
				\caption{Circuito 4 - Diagrama de Bode}
			\end{figure}
			\subsubsection{Resposta ao degrau unitário}
			\begin{figure}[!ht]
				\centering
				\includegraphics[scale=0.68]{img/1g_circ4.png}
				\caption{Circuito 4 - Resposta ao degrau unitário}
			\end{figure}
			\subsubsection{Resposta a rampa unitário}
			\begin{figure}[!ht]
				\centering
				\includegraphics[scale=0.68]{img/1h_circ4.png}
				\caption{Circuito 4 - Resposta a rampa unitária}
			\end{figure}

			\subsubsection{Resposta a onda quadrada}
			\begin{figure}[!ht]
				\centering
				\includegraphics[scale=0.71]{img/1i_circ4.png}
				\caption{Circuito 4 - Resposta a onda quadrada com $\omega = \frac{1}{8}\pi$}
			\end{figure}
			\begin{figure}[!ht]
				\centering
				\includegraphics[scale=0.71]{img/1j_circ4.png}
				\caption{Circuito 4 - Resposta ao primeiro harmônico da série de Fourier de um onda quadrada com $\omega = \frac{1}{8}\pi$}
			\end{figure}
			\begin{figure}[!ht]
				\centering
				\includegraphics[scale=0.71]{img/1k_circ4.png}
				\caption{Circuito 4 - Resposta ao terceiro harmônico da série de Fourier de um onda quadrada com $\omega = \frac{1}{8}\pi$}
			\end{figure}
			\begin{figure}[!ht]
				\centering
				\includegraphics[scale=0.71]{img/1l_circ4.png}
				\caption{Circuito 4 - Resposta ao quinto harmônico da série de Fourier de um onda quadrada com $\omega = \frac{1}{8}\pi$}
			\end{figure}
			\begin{figure}[!ht]
				\centering
				\includegraphics[scale=0.71]{img/1m_circ4.png}
				\caption{Circuito 4 - Resposta ao sétimo harmônico da série de Fourier de um onda quadrada com $\omega = \frac{1}{8}\pi$}
			\end{figure}
		\clearpage
		\subsection{Circuito 5}
			\subsubsection{Determinar a função do circuto}
			\begin{figure}[!ht]
				\centering
				\includegraphics{img/circuito5.png}
				\caption{Circuito 5}
			\end{figure}

			Esse circuito pode ser escrito  como:
			\[
			\frac{V_{in}}{R} + C\frac{\partial V_{out}}{\partial t} = 0
			\] 	\\

			Transformando esta E.D.O com Laplace utilizando o mesmo método dos circuitos passados, obtemos:

			\[
			H(S) = \frac{-1}{RCS}
			\] 	\\

			Tomando os seguintes valores para os elementos do circuito:
			\begin{itemize}
				\item $R = 1M\Omega;$
				\item $C = 50nF;$
			\end{itemize}

			Temos a seguinte equação de transferência:

			\[
			H(S) = \frac{-1}{0.05S}
			\] 	\\

			A partir dessa equação, obtemos os seguintes polos, zeros e diagrama de Bode:

			\begin{figure}[!ht]
				\centering
				\includegraphics[scale=0.75]{img/1e_circ5.png}
				\caption{Circuito 5 - Polos e Zeros}
			\end{figure}
			\newpage
			\begin{figure}[!ht]
				\centering
				\includegraphics[scale=0.70]{img/1f_circ5.png}
				\caption{Circuito 5 - Diagrama de Bode}
			\end{figure}

			Este circuito corresponde a um filtro passa baixa integrador de apenas um polo.

			\subsubsection{Resposta ao degrau unitário}
			\begin{figure}[!ht]
				\centering
				\includegraphics[scale=0.68]{img/1g_circ5.png}
				\caption{Circuito 5 - Resposta ao degrau unitário}
			\end{figure}
			\subsubsection{Resposta a rampa unitária}
			\begin{figure}[!ht]
				\centering
				\includegraphics[scale=0.68]{img/1h_circ5.png}
				\caption{Circuito 5 - Resposta a rampa unitária}
			\end{figure}

			\subsubsection{Resposta a onda quadrada}
			\begin{figure}[!ht]
				\centering
				\includegraphics[scale=0.71]{img/1i_circ5.png}
				\caption{Circuito 5 - Resposta a onda quadrada com $\omega = \frac{1}{8}\pi$}
			\end{figure}
			\begin{figure}[!ht]
				\centering
				\includegraphics[scale=0.71]{img/1j_circ5.png}
				\caption{Circuito 5 - Resposta ao primeiro harmônico da série de Fourier de um onda quadrada com $\omega = \frac{1}{8}\pi$}
			\end{figure}
			\begin{figure}[!ht]
				\centering
				\includegraphics[scale=0.71]{img/1k_circ5.png}
				\caption{Circuito 5 - Resposta ao terceiro harmônico da série de Fourier de um onda quadrada com $\omega = \frac{1}{8}\pi$}
			\end{figure}
			\begin{figure}[!ht]
				\centering
				\includegraphics[scale=0.71]{img/1l_circ5.png}
				\caption{Circuito 5 - Resposta ao quinto harmônico da série de Fourier de um onda quadrada com $\omega = \frac{1}{8}\pi$}
			\end{figure}
			\begin{figure}[!ht]
				\centering
				\includegraphics[scale=0.71]{img/1m_circ5.png}
				\caption{Circuito 5 - Resposta ao sétimo harmônico da série de Fourier de um onda quadrada com $\omega = \frac{1}{8}\pi$}
			\end{figure}

	\clearpage
	\section{Quest\~{a}o 2}

	\begin{figure}[!ht]
		\centering
        \includegraphics{img/db.png}
		\caption{Diagrama de Blocos}
	\end{figure}
	\subsection{Equações do diagrama}
	Seguindo as regras definidas no trabalho em relação aos valores de A, B, C e D para o diagrama de blocos, temos:
		\begin{itemize}
			\item a = -22;
			\item b = 7;
			\item c = 3;
			\item d = 4;
		\end{itemize}

	Pare que o circuito possua estabilidade BIBO, precisamos que o valor de A seja negativo, caso contrario, o circuito é instável.

	Com esses valores obtemos as seguintes equações para o diagrama de blocos abaixo:

		\begin{itemize}
			\item y(t) = 4u(t) + 3x(t) \textit{(Item (e) da questão 2)};
			\item B = 7u(t);
			\item C = 3x(t);
			\item D = 4u(t);
			\item $x^{'}(t) = 7u(t) - 22x(t)$ \textit{(Item (d) da questão 2)};
		\end{itemize}

	Sabendo que $x(t) = \frac{y(t) - 4u(t)}{3}$ e  $x^{'} = \frac{y^{'}(t) - 4u^{'}(t)}{3}$, obtemos a seguinte E.D.O:

	\[
	\frac{\partial y(t)}{\partial t} + 22y(t) = 4\frac{\partial y(t)}{\partial t} + 109u(t)
	\] 	\\

	Aplicando Laplace, obtemos a seguinte função de transferência:

	\[
	H(S) = \frac{Y(S)}{U(S)} = \frac{4S + 109}{S +22}
	\] 	\\

	De posse da função de transferência, podemos encontrar os seguintes polos e zeros e o diagrama de Bode:
	\begin{figure}[!ht]
		\centering
		\includegraphics{img/2b.png}
		\caption{Polos e Zeros}
	\end{figure}
	\newpage
	\begin{figure}[!ht]
		\centering
		\includegraphics[scale=0.71]{img/2c.png}
		\caption{Diagrama de Bode}
	\end{figure}
	\subsection{Resposta ao degrau unitário}
		\begin{figure}[!ht]
			\centering
			\includegraphics[scale=0.71]{img/2f.png}
			\caption{Resposta ao degrau unitário}
		\end{figure}
	\newpage
	\subsection{Resposta a rampa unitária}
		\begin{figure}[!ht]
			\centering
			\includegraphics[scale=0.71]{img/2g.png}
			\caption{Resposta a rampa unitária}
		\end{figure}

			\subsection{Resposta a onda quadrada}
			\begin{figure}[!ht]
				\centering
				\includegraphics[scale=0.71]{img/2h.png}
				\caption{Resposta a onda quadrada com $\omega = \frac{1}{4}\pi$}
			\end{figure}
			\begin{figure}[!ht]
				\centering
				\includegraphics[scale=0.71]{img/2i.png}
				\caption{Resposta ao primeiro harmônico da série de Fourier de um onda quadrada com $\omega = \frac{1}{2}\pi$}
			\end{figure}
			\begin{figure}[!ht]
				\centering
				\includegraphics[scale=0.71]{img/2j.png}
				\caption{Resposta ao terceiro harmônico da série de Fourier de um onda quadrada com $\omega = \frac{1}{2}\pi$}
			\end{figure}
			\begin{figure}[!ht]
				\centering
				\includegraphics[scale=0.71]{img/2k.png}
				\caption{Resposta ao quinto harmônico da série de Fourier de um onda quadrada com $\omega = \frac{1}{2}\pi$}
			\end{figure}
			\begin{figure}[!ht]
				\centering
				\includegraphics[scale=0.71]{img/2l.png}
				\caption{Resposta ao sétimo harmônico da série de Fourier de um onda quadrada com $\omega = \frac{1}{2}\pi$}
			\end{figure}
			\clearpage

	\section{Quest\~{a}o 3}
		Resolva as questões para os sistemas descritos pelas seguintes funções de transferência:
		\[
		H(S)= \frac{1 + \alpha S}{S^{2} + 2S +  2}
		\] 	\\
		\[
		H(S)= \frac{S + 10^{4}}{S^{2} + 2\beta S +  100}
		\] 	\\

		\subsection{E.D.O dos sistemas}

			\subsubsection{Sistema 1}
				\[
				X(S)[1+ \alpha S] = Y(S)[S^{2} + 2S + 2] \Rightarrow
				\] 	\\
				\[
				\alpha \frac{\partial x(t)}{\partial t} + x(t) = \frac{\partial^{2}y(t)}{\partial^{2}t} + 2\frac{\partial^y(t)}{\partial t} + 2y(t)
				\] 	\\
			\subsubsection{Sistema 2}
				\[
				X(S)[S + 10^{4}] = Y(S)[S^{2} + 2\beta S +  100] \Rightarrow
				\] 	\\
				\[
				\frac{\partial x(t)}{\partial t} + 10^{4}x(t) = \frac{\partial^{2}y(t)}{\partial^{2}t} + 2 \beta\frac{\partial y(t)}{\partial t} + 100y(t)
				\] 	\\
				\newpage
		\subsection{Polos e zeros}
			\subsubsection{Variando em $\alpha$}
			\begin{figure}[!ht]
				\centering
				\includegraphics[scale=0.5]{img/3b_alfa.png}
				\caption{Polos e Zeros variando em $\alpha$}
			\end{figure}
			\subsubsection{Variando em $\beta$}
			\begin{figure}[!ht]
				\centering
				\includegraphics[scale=0.55]{img/3b_beta.png}
				\caption{Polos e Zeros variando em $\beta$}
			\end{figure}
		\subsection{Diagrama de Bode}
			\subsubsection{Variando em $\alpha$}
			\begin{figure}[!ht]
				\centering
				\includegraphics[scale=0.42]{img/3c_alfa.png}
				\caption{Diagrama de Bode variando em $\alpha$}
			\end{figure}
			\subsubsection{Variando em $\beta$}
			\begin{figure}[!ht]
				\centering
				\includegraphics[scale=0.46]{img/3c_beta.png}
				\caption{Diagrama de Bode variando em $\beta$}
			\end{figure}
		\subsection{Resposta ao Degrau Unitário}
			\subsubsection{Variando em $\alpha$}
			\begin{figure}[!ht]
				\centering
				\includegraphics[scale=0.45]{img/3d_alfa.png}
				\caption{Resposta ao Degrau Unitário variando em $\alpha$}
			\end{figure}
			\subsubsection{Variando em $\beta$}
			\begin{figure}[!ht]
				\centering
				\includegraphics[scale=0.5]{img/3d_beta.png}
				\caption{Resposta ao Degrau Unitário variando em $\beta$}
			\end{figure}
    	\newpage
		\subsection{Resposta a Rampa Unitária}
		\subsubsection{Variando em $\alpha$}
		\begin{figure}[!ht]
			\centering
			\includegraphics[scale=0.5]{img/3e_alfa.png}
			\caption{Resposta a rampa Unitária variando em $\alpha$}
		\end{figure}
		\subsubsection{Variando em $\beta$}
		\begin{figure}[!ht]
			\centering
			\includegraphics[scale=0.45]{img/3e_beta.png}
			\caption{Resposta a rampa Unitária variando em $\beta$}
		\end{figure}
		\subsection{Resposta a onda quadrada}
			\subsubsection{Variando em $\alpha$}
			\begin{figure}[!ht]
				\centering
				\includegraphics[scale=0.4]{img/3f_alfa.png}
				\caption{Resposta a onda quadrada com $\omega = \frac{1}{8}\pi$}
			\end{figure}
			\begin{figure}[!ht]
				\centering
				\includegraphics[scale=0.5]{img/3g_alfa.png}
				\caption{Resposta ao primeiro harmônico da série de Fourier de um onda quadrada com $\omega = \frac{1}{4}\pi$}
			\end{figure}
			\begin{figure}[!ht]
				\centering
				\includegraphics[scale=0.5]{img/3h_alfa.png}
				\caption{Resposta ao terceiro harmônico da série de Fourier de um onda quadrada com $\omega = \frac{1}{4}\pi$}
			\end{figure}
			\begin{figure}[!ht]
				\centering
				\includegraphics[scale=0.5]{img/3i_alfa.png}
				\caption{Resposta ao quinto harmônico da série de Fourier de um onda quadrada com $\omega = \frac{1}{4}\pi$}
			\end{figure}
			\begin{figure}[!ht]
				\centering
				\includegraphics[scale=0.57]{img/3j_alfa.png}
				\caption{Resposta ao sétimo harmônico da série de Fourier de um onda quadrada com $\omega = \frac{1}{4}\pi$}
			\end{figure}
			\clearpage
			\subsubsection{Variando em $\beta$}
			\begin{figure}[!ht]
				\centering
				\includegraphics[scale=0.46]{img/3f_beta.png}
				\caption{Resposta a onda quadrada com $\omega = \frac{1}{8}\pi$}
			\end{figure}
			\begin{figure}[!ht]
				\centering
				\includegraphics[scale=0.57]{img/3g_beta.png}
				\caption{Resposta ao primeiro harmônico da série de Fourier de um onda quadrada com $\omega = \frac{1}{4}\pi$}
			\end{figure}
			\begin{figure}[!ht]
				\centering
				\includegraphics[scale=0.52]{img/3h_beta.png}
				\caption{Resposta ao terceiro harmônico da série de Fourier de um onda quadrada com $\omega = \frac{1}{4}\pi$}
			\end{figure}
			\begin{figure}[!ht]
				\centering
				\includegraphics[scale=0.48]{img/3i_beta.png}
				\caption{Resposta ao quinto harmônico da série de Fourier de um onda quadrada com $\omega = \frac{1}{4}\pi$}
			\end{figure}
			\begin{figure}[!ht]
				\centering
				\includegraphics[scale=0.5]{img/3j_beta.png}
				\caption{Resposta ao sétimo harmônico da série de Fourier de um onda quadrada com $\omega = \frac{1}{4}\pi$}
			\end{figure}

	\subsection{Resposta a cossenoides}
	\subsubsection{Variando frequências nos valores de $\alpha$}
			\begin{figure}[!ht]
				\centering
				\includegraphics[scale=0.45]{img/3k_alfa1.png}
				\caption{Resposta para $\alpha$ = 0.001 em frequências variantes}
			\end{figure}
			\begin{figure}[!ht]
				\centering
				\includegraphics[scale=0.48]{img/3k_alfa2.png}
				\caption{Resposta para $\alpha$ = 0.01 em frequências variantes}
			\end{figure}
			\begin{figure}[!ht]
				\centering
				\includegraphics[scale=0.48]{img/3k_alfa3.png}
				\caption{Resposta para $\alpha$ = 0.1 em frequências variantes}
			\end{figure}
			\begin{figure}[!ht]
				\centering
				\includegraphics[scale=0.52]{img/3k_alfa4.png}
				\caption{Resposta para $\alpha$ = 1 em frequências variantes}
			\end{figure}
			\begin{figure}[!ht]
				\centering
				\includegraphics[scale=0.52]{img/3k_alfa5.png}
				\caption{Resposta para $\alpha$ = 10 em frequências variantes}
			\end{figure}
			\begin{figure}[!ht]
				\centering
				\includegraphics[scale=0.48]{img/3k_alfa6.png}
				\caption{Resposta para $\alpha$ = 100 em frequências variantes}
			\end{figure}
			\begin{figure}[!ht]
				\centering
				\includegraphics[scale=0.49]{img/3k_alfa7.png}
				\caption{Resposta para $\alpha$ = 1000 em frequências variantes}
			\end{figure}
			\clearpage
	\subsubsection{Variando frequências nos valores de $\beta$}
			\begin{figure}[!ht]
				\centering
				\includegraphics[scale=0.58]{img/3k_beta1.png}
				\caption{Resposta para $\beta$ = 0.001 em frequências variantes}
			\end{figure}
			\begin{figure}[!ht]
				\centering
				\includegraphics[scale=0.53]{img/3k_beta2.png}
				\caption{Resposta para $\beta$ = 0.01 em frequências variantes}
			\end{figure}
			\begin{figure}[!ht]
				\centering
				\includegraphics[scale=0.49]{img/3k_beta3.png}
				\caption{Resposta para $\beta$ = 0.1 em frequências variantes}
			\end{figure}
			\begin{figure}[!ht]
				\centering
				\includegraphics[scale=0.57]{img/3k_beta4.png}
				\caption{Resposta para $\beta$ = 1 em frequências variantes}
			\end{figure}
			\begin{figure}[!ht]
				\centering
				\includegraphics[scale=0.57]{img/3k_beta5.png}
				\caption{Resposta para $\beta$ = 10 em frequências variantes}
			\end{figure}
	\clearpage		
	\newpage
	\section{Questão 4}\label{q4}
		\subsection{Circuito III}
			\subsubsection{Análise da resposta na Frequência de Corte}
			\begin{figure}[!ht]
				\centering
				\includegraphics[scale=0.06, angle=180]{img/w.jpg}
				\caption{Análise da resposta na Frequência de corte}
			\end{figure}		
			\subsubsection{Análise da resposta na Frequência em $0.1w_{c}$}
			\begin{figure}[!ht]
				\centering
				\includegraphics[scale=0.06, angle=180]{img/01w.jpg}
				\caption{Análise da resposta na Frequência de corte uma década antes}
			\end{figure}		
			\subsubsection{Análise da resposta na Frequência em $10w_{c}$}
			\begin{figure}[!ht]
				\centering
				\includegraphics[scale=0.06, angle=180]{img/10w.jpg}
				\caption{Análise da resposta na Frequência de corte uma década depois}
			\end{figure}
			\subsubsection{Harmônicos de Fourier}
			\begin{figure}[!ht]
				\centering
				\includegraphics[scale=0.06, angle=180]{img/harm1.jpg}
				\caption{Primeiro harmônico}
			\end{figure}					
			\begin{figure}[!ht]
				\centering
				\includegraphics[scale=0.08, angle=180]{img/harm3.jpg}
				\caption{Terceiro harmônico}
			\end{figure}					
			\begin{figure}[!ht]
				\centering
				\includegraphics[scale=0.08, angle=180]{img/harm5.jpg}
				\caption{Quinto harmônico}
			\end{figure}					
			\begin{figure}[!ht]
				\centering
				\includegraphics[scale=0.06, angle=180]{img/harm7.jpg}
				\caption{Sétimo harmônico}
			\end{figure}		

			\clearpage
			\newpage
			\subsubsection{Respostas para diferentes frequências}
			\begin{figure}[!ht]
				\centering
				\includegraphics[scale=0.07, angle=180]{img/100h.jpg}
				\caption{Cossenóide 100Hz}
			\end{figure}					
			\begin{figure}[!ht]
				\centering
				\includegraphics[scale=0.07, angle=180]{img/1k.jpg}
				\caption{Cossenóide 1kHz}
			\end{figure}					
			\begin{figure}[!ht]
				\centering
				\includegraphics[scale=0.08, angle=180]{img/10k.jpg}
				\caption{Cossenóide 10kHz}
			\end{figure}						
	\newpage
	\section{Conclusão}

	Boa parte desse trabalho foi reaproveitado do meu trabalho feito no período passado, com exceção das seções \ref{conhecimentos} \ref{q1} e \ref{q4}. Em especial adicionei os conhecimentos utilizados para resolver o trabalho explicando todos os artifícios para resolver cada um dos itens solicitados e remodelei os circuitos com valores realísticos (Removi os indutores de 1H :) é muito fio para enrolar!!) .

	O trabalho foi bastante importante para fixar alguns conceitos aprendidos na sala, dos quais eu posso listar como mais importante:

		\begin{itemize}
			\item Analise dos circuitos;
			\item Transformar E.D.O do circuito em funções de transferência através da transformada de Laplace;
			\item Verificar como os polos e o zeros influenciam na pratica a plotagem do diagrama de bode;
			\item Entender como o valor dos componentes influenciam nos polos e zeros e nas frequências de filtragem;
			\item Compreender superficialmente o funcionamento de filtros;
			\item Representação de circuitos em diagrama de blocos;
			\item Encontrar a reposta do sistema para diferentes sinais através de sua função de transferência.
			\item Verificar que a resposta ao somatório dos harmônicos de fourier se aproxima do sinal normal conforme o numero de harmônicos crescem;
			\item Aprender a realizar simulações no MatLab e no Octave;
			\item Montagem de circuito na prática;
		\end{itemize}

	\newpage
	\section{Referências}

	[1] https://en.wikipedia.org/wiki/Buffer\_amplifier; \\

	[2] https://en.wikipedia.org/wiki/Electronic\_filter;\\

	[3] https://en.wikipedia.org/wiki/Low-pass\_filter;\\

	[4] https://en.wikipedia.org/wiki/Band-pass\_filter;\\

	[5] https://en.wikipedia.org/wiki/Butterworth\_filter;\\

	[6] https://en.wikipedia.org/wiki/Sallen-Key\_topology;\\

	[7] https://en.wikipedia.org/wiki/Integrator;\\

	[8] https://en.wikibooks.org/wiki/Signals\_and\_Systems;\\

	[9] http://www.lps.ufrj.br/\~natmourajr/EEL350/2016\_01/slides\_SL1.pdf; \\

	[10] B. P. Lathi, Linear Systems and Signals. Oxford, UK: Oxford University Press, 2nd ed., 2009. \\

	[11] A. V. Oppenheim, A. S. Willsky, and S. H. Nawab, Signals and Systems (2Nd Ed.). Upper Saddle River, NJ, USA: Prentice-Hall, Inc., 1996.

\end{document}