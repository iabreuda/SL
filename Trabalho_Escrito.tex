\documentclass[a4paper, 12pt]{article}

\usepackage[portuges]{babel}
\usepackage[utf8]{inputenc}
\usepackage{amsmath}
\usepackage{indentfirst}
\usepackage{blindtext}
\usepackage{graphicx}
\usepackage[hidelinks]{hyperref}

\author{Igor Abreu da Silva}

\title{Trabalho Final de Sistemas Lineares I}

\begin{document}
	
	\begin{titlepage}
		\begin{center}
			\huge{Universidade Federal do Rio de Janeiro}
			\vspace{95pt}

			\large{Trabalho Final de Sistemas Lineares I}
			\vspace{160pt}
		\end{center}
		
		\begin{flushleft}
			\begin{tabbing}
				Alunos\qquad\qquad\= Igor Abreu da Silva\\
				DRE\> 112053874 \\
				Curso\> Engenharia Eletrônica \\
				Turma\> 2016/1 \\
				Professor\> Natanael Nunes de Moura Junior \\
			
			\end{tabbing}
			
		\end{flushleft}
		
		\begin{center}
			\vspace{\fill}
			Rio de Janeiro, 15 de Julho de 2016
		\end{center}
	\end{titlepage}


	\newpage
	\tableofcontents
	\listoffigures
	\thispagestyle{empty}
	
	\newpage
	\pagenumbering{arabic}
	
	\section{Quest\~{a}o 1}
		\subsection{Circuito 1}
			Nesta sessão será resolvida toda a parte necessária para encontra a função/utilidade de cada um dos circuitos. Analisaremos todos os pontos correspondentes aos itens \textbf{(a)}, \textbf{(b)}, \textbf{(c)}, \textbf{(d)}, \textbf{(e)} e \textbf{(f)} do trabalho final. 
			
			
			Serão assumidos aqui que os sistemas encontram-se o zerados no instante $t = 0^{-}$.
			\subsubsection{Determinar a função do circuto}
			

			\begin{figure}[!ht]
				\centering
				\includegraphics{img/circuito1.png}
				\caption{Circuito 1}	
			\end{figure}			
			Podemos modelar o circuito 1 em relação ao nó após R1. Teríamos a seguinte equação: \\
			\[
				\frac{V_{in} - V_{out}}{R1} - \frac{V_{out}}{R2} - \frac{C\partial V_{out}}{\partial t} - \frac{1}{L} \int V_{out}{\partial t} = 0
			\] 	\\
			Para encontrarmos a E.D.O do circuito, vamos derivar toda esta expressão e separar $V_{out}$ e $V_{in}$, encontrando a seguinte relação:
			\[
				\frac{\partial V_{in}}{\partial t} \left(\frac{1}{R_{1}}\right) = \frac{C \partial^{2} V_{out}}{\partial t^{2}} + \frac{\partial V_{out}}{\partial t} \left(\frac{1}{R_{1}} + \frac{1}{R_{2}} \right) + \frac{V_{out}}{L}
			\] 	\\			
			Em posse da E.D.O, utilizaremos Laplace para encontrar a função de Transferência do Circuito.
			\[
				X(S) \left(\frac{1}{R_{1}}\right) = Y(S)\left(S^{2}C + S \left(\frac{1}{R_{1}} + \frac{1}{R_{2}}\right) + \frac{1}{L} \right) \Rightarrow
			\] 	\\	
			\[
				H(S) = \frac{Y(S)}{X(S)} = \frac{SR_{2}L}{S^{2}\left(R_{1}R_{2}LC\right) + S\left(R_{1}L + R_{2}L\right) + R_{1}R_{2}}
			\] 	\\		
			
			Afim de facilitar os cálculos, tomaremos os seguintes valores para cada elemento do circuito:
			 \begin{itemize}
			 	\item $R_{1} = 10\Omega;$
			 	\item $R_{2} = 100\Omega;$
			 	\item $C = 1F;$
			 	\item $L = 1H;$
			 \end{itemize}		
			 
			Apos aplicar os valores comercias em H(S), temos:
			\[
			H(S) = \frac{100S}{1000S^{2} + 110S + 110}
			\] 	\\					
			
			Utilizando essa função no MatLab para encontrar os polos (quando se zera o denominador), zeros (quando se zera o numerador) e o diagrama de Bode, obtemos o seguintes gráficos:
			
			\begin{figure}[!ht]
				\centering
				\includegraphics[scale=0.7]{img/1e_circ1.png}
				\caption{Circuito 1 - Polos e Zeros}	
			\end{figure}	
			
			\begin{figure}[!ht]
				\centering
				\includegraphics[scale=0.9]{img/1f_circ1.png}
				\caption{Circuito 1 - Diagrama de Bode}	
			\end{figure}						
											
			Analisando-se este circuito, pode-se afirmar que o mesmo é um filtro passa faixa operando na largura de banda de aproximadamente 0.11 rad/sec em um intervalo [0.28, 0.39] rad/sec.
			\newpage
			\subsubsection{Resposta ao degrau unitário}
			\begin{figure}[!ht]
				\centering
				\includegraphics[scale=0.71]{img/1g_circ1.png}
				\caption{Circuito 1 - Resposta ao degrau unitário}	
			\end{figure}					
			\subsubsection{Resposta a rampa unitário}
			\begin{figure}[!ht]
				\centering
				\includegraphics[scale=0.72]{img/1h_circ1.png}
				\caption{Circuito 1 - Resposta a rampa unitária}	
			\end{figure}		

			\subsubsection{Resposta a onda quadrada}
			\begin{figure}[!ht]
				\centering
				\includegraphics[scale=0.71]{img/1i_circ1.png}
				\caption{Circuito 1 - Resposta a onda quadrada com $\omega = \frac{1}{8}\pi$}	
			\end{figure}			
			\begin{figure}[!ht]
				\centering
				\includegraphics[scale=0.71]{img/1j_circ1.png}
				\caption{Circuito 1 - Resposta ao primeiro harmônico da série de Fourier de um onda quadrada com $\omega = \frac{1}{8}\pi$}	
			\end{figure}		
			\begin{figure}[!ht]
				\centering
				\includegraphics[scale=0.71]{img/1k_circ1.png}
				\caption{Circuito 1 - Resposta ao terceiro harmônico da série de Fourier de um onda quadrada com $\omega = \frac{1}{8}\pi$}	
			\end{figure}			
			\begin{figure}[!ht]
				\centering
				\includegraphics[scale=0.71]{img/1l_circ1.png}
				\caption{Circuito 1 - Resposta ao quinto harmônico da série de Fourier de um onda quadrada com $\omega = \frac{1}{8}\pi$}	
			\end{figure}				
			\begin{figure}[!ht]
				\centering
				\includegraphics[scale=0.71]{img/1m_circ1.png}
				\caption{Circuito 1 - Resposta ao sétimo harmônico da série de Fourier de um onda quadrada com $\omega = \frac{1}{8}\pi$}	
			\end{figure}
		\newpage
		\subsection{Circuito 2}
			\subsubsection{Determinar a função do circuto}
			\begin{figure}[!hb]
				\centering
				\includegraphics{img/circuito2.png}
				\caption{Circuito 2}	
			\end{figure}			
			
			Para modelarmos utilizaremos as seguintes equações:
			\[				
				 I_{1} = I_{in} - I_{out}
			\] 	\\	
			\[	
				 R_{2}I_{out} + \frac{L \partial I_{out}}{\partial t} - R_{1}I_{1} + \frac{1}{C}\int I_{out}\partial t = 0
			\] 	\\	
			
			Substituindo $I_{1}$ para colocarmos a equação em função de $I_{in}$ e $I_{out}$ e derivando-a para removermos a Integral, temos a E.D.O:
			
			\[	
				\frac{\partial I_{in}}{\partial t}\left(R_{1}\right) = \frac{\partial^{2}I_{out}}{\partial t^{2}}\left(L\right) + \frac{\partial I_{out}}{\partial t}\left(R_{1} + R_{2}\right) + \frac{I_{out}}{C}
			\] 	\\
						
			Transformando essa E.D.O em Laplace, obtemos:
			
			\[	
				X(S)\left(SR_{1}\right) = Y(S)\left(S^{2} + S\left(R_{1} +  R_{2}\right) + \frac{1}{C}\right) \Rightarrow
			\] 	\\			
			\[
			H(S) = \frac{Y(S)}{X(S)} = \frac{S\left(R_{1}C\right)}{S^{2}\left(LC\right) + S\left(R_{1}C + R_{2}C\right) + 1}
			\] 	\\					
			
			Escolhendo os seguintes valores para cada elemento do circuito:
			\begin{itemize}
				\item $R_{1} = 10\Omega;$
				\item $R_{2} = 100\Omega;$
				\item $C = 1F;$
				\item $L = 1H;$
			\end{itemize}	
							
			Encontramos a seguinte função de transferência:
			\[
				H(S) = \frac{10S}{S^{2} + 110S + 1}
			\] 	\\				
			A partir dessa função obtemos os seguintes polos, zeros e diagrama de Bode:
			\begin{figure}[!ht]
				\centering
				\includegraphics[scale=0.8]{img/1e_circ2.png}
				\caption{Circuito 2 - Polos e Zeros}	
			\end{figure}	

			\begin{figure}[!ht]
				\centering
				\includegraphics[scale=0.7]{img/1f_circ2.png}
				\caption{Circuito 2 - Diagrama de Bode}	
			\end{figure}			
			\newpage		
			Assim como o circuito da figura 1, temos também um filtro passa faixa que opera nas faixas entre 0.01 rad/seg e 86.5 rad/seg
			
			\subsubsection{Resposta ao degrau unitário}
			\begin{figure}[!ht]
				\centering
				\includegraphics[scale=0.71]{img/1g_circ2.png}
				\caption{Circuito 2 - Resposta ao degrau unitário}	
			\end{figure}					
			\subsubsection{Resposta a rampa unitário}
			\begin{figure}[!ht]
				\centering
				\includegraphics[scale=0.72]{img/1h_circ2.png}
				\caption{Circuito 2 - Resposta a rampa unitária}	
			\end{figure}		
			
			\subsubsection{Resposta a onda quadrada}
			\begin{figure}[!ht]
				\centering
				\includegraphics[scale=0.71]{img/1i_circ2.png}
				\caption{Circuito 2 - Resposta a onda quadrada com $\omega = \frac{1}{8}\pi$}	
			\end{figure}			
			\begin{figure}[!ht]
				\centering
				\includegraphics[scale=0.71]{img/1j_circ2.png}
				\caption{Circuito 2 - Resposta ao primeiro harmônico da série de Fourier de um onda quadrada com $\omega = \frac{1}{8}\pi$}	
			\end{figure}		
			\begin{figure}[!ht]
				\centering
				\includegraphics[scale=0.71]{img/1k_circ2.png}
				\caption{Circuito 2 - Resposta ao terceiro harmônico da série de Fourier de um onda quadrada com $\omega = \frac{1}{8}\pi$}	
			\end{figure}			
			\begin{figure}[!ht]
				\centering
				\includegraphics[scale=0.71]{img/1l_circ2.png}
				\caption{Circuito 2 - Resposta ao quinto harmônico da série de Fourier de um onda quadrada com $\omega = \frac{1}{8}\pi$}	
			\end{figure}		
		    \clearpage
			\begin{figure}[!ht]
				\centering
				\includegraphics[scale=0.71]{img/1m_circ2.png}
				\caption{Circuito 2 - Resposta ao sétimo harmônico da série de Fourier de um onda quadrada com $\omega = \frac{1}{8}\pi$}	
			\end{figure}			
		\subsection{Circuito 3}
			\subsubsection{Determinar a função do circuto}
			\begin{figure}[!ht]
				\centering
				\includegraphics{img/circuito3.png}
				\caption{Circuito 3}	
			\end{figure}			
			
			Este circuito, também conhecido como topologia de Sallen-Key, sabendo que o AmpOp possui impedância infinita em sua entrada, que $V^{-} = V^{+}$, que $V^{-} = V_{out}$ e chamando $V_{a}$ da tensão que passa por $C_{1}$, obtemos:
				
			\[
			V_{a} = V_{out} + R_{2}C_{2}\frac{\partial V_{out}}{\partial t}
			\] 	\\					
			
			Utilizando a lei dos nós entre $R_{1}$ e $R_{2}$ e já substituindo $V_{a}$ por $V_{out}$ temos:
			
			\[
			\frac{V_{in}}{R_{1}} = R_{2}C_{1}C_{2}\frac{\partial^{2} V_{out}}{\partial t^{2}} + \left(C_{2} + \frac{R_{2}C_{2}}{R_{1}}\right)\frac{\partial V_{out}}{\partial t} +  \frac{V_{out}}{R_{1}}
			\] 	\\			
			
			Com esta E.D.O, podemos encontrar a seguinte função de transferência utilizando o mesmo método empregado nos circuitos anteriores, com isso temos:
			
			\[
			H(S) = \frac{1}{S^{2}\left(R_{1}R_{2}C_{1}C_{2}\right) + S\left(R_{1}C_{2} + R_{2}C_{2}\right) + 1}
			\] 	\\					
			
			Utilizando os valores para cada elemento do circuito:
			\begin{itemize}
				\item $R_{1} = 10\Omega;$
				\item $R_{2} = 100\Omega;$
				\item $C_{1} = 2F;$
				\item $C_{2} = 1F;$
			\end{itemize}				

			Encontramos a seguinte função de transferência:
			\[
			H(S) = \frac{1}{2000S^{2} + 110S + 1}
			\] 	\\				
			
			Que nos gera os seguintes polos, zeros e diagrama de Bode:	
			\begin{figure}[!ht]
				\centering
				\includegraphics[scale=0.7]{img/1e_circ3.png}
				\caption{Circuito 3 - Polos e Zeros}	
			\end{figure}	
			\newpage
			\begin{figure}[!ht]
				\centering
				\includegraphics[scale=0.7]{img/1f_circ3.png}
				\caption{Circuito 3 - Diagrama de Bode}	
			\end{figure}			
			Pela a analise do diagrama de Bode, pode-se afirmar que esse circuito é um filtro passa alta com frequência no seu menor polo de 0.01 rad/sec.
			\subsubsection{Resposta ao degrau unitário}
			\begin{figure}[!ht]
				\centering
				\includegraphics[scale=0.71]{img/1g_circ3.png}
				\caption{Circuito 3 - Resposta ao degrau unitário}	
			\end{figure}					
			\subsubsection{Resposta a rampa unitário}
			\begin{figure}[!ht]
				\centering
				\includegraphics[scale=0.72]{img/1h_circ3.png}
				\caption{Circuito 3 - Resposta a rampa unitária}	
			\end{figure}		
			
			\subsubsection{Resposta a onda quadrada}
			\begin{figure}[!ht]
				\centering
				\includegraphics[scale=0.71]{img/1i_circ3.png}
				\caption{Circuito 3 - Resposta a onda quadrada com $\omega = \frac{1}{8}\pi$}	
			\end{figure}			
			\begin{figure}[!ht]
				\centering
				\includegraphics[scale=0.71]{img/1j_circ3.png}
				\caption{Circuito 3 - Resposta ao primeiro harmônico da série de Fourier de um onda quadrada com $\omega = \frac{1}{8}\pi$}	
			\end{figure}		
			\begin{figure}[!ht]
				\centering
				\includegraphics[scale=0.71]{img/1k_circ3.png}
				\caption{Circuito 3 - Resposta ao terceiro harmônico da série de Fourier de um onda quadrada com $\omega = \frac{1}{8}\pi$}	
			\end{figure}			
			\begin{figure}[!ht]
				\centering
				\includegraphics[scale=0.71]{img/1l_circ3.png}
				\caption{Circuito 3 - Resposta ao quinto harmônico da série de Fourier de um onda quadrada com $\omega = \frac{1}{8}\pi$}	
			\end{figure}		
			\clearpage
			\begin{figure}[!ht]
				\centering
				\includegraphics[scale=0.71]{img/1m_circ3.png}
				\caption{Circuito 3 - Resposta ao sétimo harmônico da série de Fourier de um onda quadrada com $\omega = \frac{1}{8}\pi$}	
			\end{figure}				
		\subsection{Circuito 4}
			\subsubsection{Determinar a função do circuto}
			\begin{figure}[!ht]
				\centering
				\includegraphics{img/circuito4.png}
				\caption{Circuito 4}	
			\end{figure}		
			
			Esse circuito, conhecido como buffer, é utilizado como um isolador. Como $V_{in}$ é igual a $V_{out}$, sua função de transferência  H(S) = 1. Não existem polos nem zeros para esse circuito e seu diagrama de Bode permanece em 0.
			\begin{figure}[!ht]
				\centering
				\includegraphics[scale=0.7]{img/1f_circ4.png}
				\caption{Circuito 4 - Diagrama de Bode}	
			\end{figure}	
			\subsubsection{Resposta ao degrau unitário}
			\begin{figure}[!ht]
				\centering
				\includegraphics[scale=0.71]{img/1g_circ4.png}
				\caption{Circuito 4 - Resposta ao degrau unitário}	
			\end{figure}					
			\subsubsection{Resposta a rampa unitário}
			\begin{figure}[!ht]
				\centering
				\includegraphics[scale=0.72]{img/1h_circ4.png}
				\caption{Circuito 4 - Resposta a rampa unitária}	
			\end{figure}		
			
			\subsubsection{Resposta a onda quadrada}
			\begin{figure}[!ht]
				\centering
				\includegraphics[scale=0.71]{img/1i_circ4.png}
				\caption{Circuito 4 - Resposta a onda quadrada com $\omega = \frac{1}{8}\pi$}	
			\end{figure}			
			\begin{figure}[!ht]
				\centering
				\includegraphics[scale=0.71]{img/1j_circ4.png}
				\caption{Circuito 4 - Resposta ao primeiro harmônico da série de Fourier de um onda quadrada com $\omega = \frac{1}{8}\pi$}	
			\end{figure}		
			\begin{figure}[!ht]
				\centering
				\includegraphics[scale=0.71]{img/1k_circ4.png}
				\caption{Circuito 4 - Resposta ao terceiro harmônico da série de Fourier de um onda quadrada com $\omega = \frac{1}{8}\pi$}	
			\end{figure}			
			\begin{figure}[!ht]
				\centering
				\includegraphics[scale=0.71]{img/1l_circ4.png}
				\caption{Circuito 4 - Resposta ao quinto harmônico da série de Fourier de um onda quadrada com $\omega = \frac{1}{8}\pi$}	
			\end{figure}		
			\begin{figure}[!ht]
				\centering
				\includegraphics[scale=0.71]{img/1m_circ4.png}
				\caption{Circuito 4 - Resposta ao sétimo harmônico da série de Fourier de um onda quadrada com $\omega = \frac{1}{8}\pi$}	
			\end{figure}					
		\clearpage		
		\subsection{Circuito 5}
			\subsubsection{Determinar a função do circuto}
			\begin{figure}[!ht]
				\centering
				\includegraphics{img/circuito5.png}
				\caption{Circuito 5}	
			\end{figure}		
			
			Esse circuito pode ser escrito  como:
			\[
			\frac{V_{in}}{R} + C\frac{\partial V_{out}}{\partial t} = 0
			\] 	\\			
			
			Transformando esta E.D.O com Laplace utilizando o mesmo método dos circuitos passados, obtemos:
			
			\[
			H(S) = \frac{-1}{RCS}
			\] 	\\				
			
			Tomando os seguintes valores para os elementos do circuito:
			\begin{itemize}
				\item $R = 10\Omega;$
				\item $C = 1F;$
			\end{itemize}		
			
			Temos a seguinte equação de transferência:	
			
			\[
			H(S) = \frac{-1}{10S}
			\] 	\\					
			
			A partir dessa equação, obtemos os seguintes polos, zeros e diagrama de Bode:	
			
			\begin{figure}[!ht]
				\centering
				\includegraphics[scale=0.75]{img/1e_circ5.png}
				\caption{Circuito 5 - Polos e Zeros}	
			\end{figure}	
			\newpage
			\begin{figure}[!ht]
				\centering
				\includegraphics[scale=0.78]{img/1f_circ5.png}
				\caption{Circuito 5 - Diagrama de Bode}	
			\end{figure}				
			
			Este circuito corresponde a um filtro passa baixa integrador de apenas um polo.	
			
			\subsubsection{Resposta ao degrau unitário}
			\begin{figure}[!ht]
				\centering
				\includegraphics[scale=0.71]{img/1g_circ5.png}
				\caption{Circuito 5 - Resposta ao degrau unitário}	
			\end{figure}					
			\subsubsection{Resposta a rampa unitária}
			\begin{figure}[!ht]
				\centering
				\includegraphics[scale=0.72]{img/1h_circ5.png}
				\caption{Circuito 5 - Resposta a rampa unitária}	
			\end{figure}		
			
			\subsubsection{Resposta a onda quadrada}
			\begin{figure}[!ht]
				\centering
				\includegraphics[scale=0.71]{img/1i_circ5.png}
				\caption{Circuito 5 - Resposta a onda quadrada com $\omega = \frac{1}{8}\pi$}	
			\end{figure}			
			\begin{figure}[!ht]
				\centering
				\includegraphics[scale=0.71]{img/1j_circ5.png}
				\caption{Circuito 5 - Resposta ao primeiro harmônico da série de Fourier de um onda quadrada com $\omega = \frac{1}{8}\pi$}	
			\end{figure}		
			\begin{figure}[!ht]
				\centering
				\includegraphics[scale=0.71]{img/1k_circ5.png}
				\caption{Circuito 5 - Resposta ao terceiro harmônico da série de Fourier de um onda quadrada com $\omega = \frac{1}{8}\pi$}	
			\end{figure}			
			\begin{figure}[!ht]
				\centering
				\includegraphics[scale=0.71]{img/1l_circ5.png}
				\caption{Circuito 5 - Resposta ao quinto harmônico da série de Fourier de um onda quadrada com $\omega = \frac{1}{8}\pi$}	
			\end{figure}		
			\begin{figure}[!ht]
				\centering
				\includegraphics[scale=0.71]{img/1m_circ5.png}
				\caption{Circuito 5 - Resposta ao sétimo harmônico da série de Fourier de um onda quadrada com $\omega = \frac{1}{8}\pi$}	
			\end{figure}						
									
	\clearpage
	\section{Quest\~{a}o 2}
	
	\begin{figure}[!ht]
		\centering
		\includegraphics{img/db.png}
		\caption{Diagrama de Blocos}	
	\end{figure}	
	\subsection{Equações do diagrama}
	Seguindo as regras definidas no trabalho em relação aos valores de A, B, C e D para o diagrama de blocos, temos:
		\begin{itemize}
			\item a = -22;
			\item b = 7;
			\item c = 3;
			\item d = 4;
		\end{itemize}		
		
	Pare que o circuito possua estabilidade BIBO, precisamos que o valor de A seja negativo, caso contrario, o circuito é instável.
	
	Com esses valores obtemos as seguintes equações para o diagrama de blocos abaixo:
	
		\begin{itemize}
			\item y(t) = 4u(t) + 3x(t) \textit{(Item (e) da questão 2)};
			\item B = 7u(t);
			\item C = 3x(t);
			\item D = 4u(t);
			\item $x^{'}(t) = 7u(t) - 22x(t)$ \textit{(Item (d) da questão 2)};
		\end{itemize}	
		
	Sabendo que $x(t) = \frac{y(t) - 4u(t)}{3}$ e  $x^{'} = \frac{y^{'}(t) - 4u^{'}(t)}{3}$, obtemos a seguinte E.D.O:
	
	\[
	\frac{\partial y(t)}{\partial t} + 22y(t) = 4\frac{\partial y(t)}{\partial t} + 109u(t)
	\] 	\\	
	
	Aplicando Laplace, obtemos a seguinte função de transferência:
	
	\[
	H(S) = \frac{Y(S)}{U(S)} = \frac{4S + 109}{S +22}
	\] 	\\		

	De posse da função de transferência, podemos encontrar os seguintes polos e zeros e o diagrama de Bode:
	\begin{figure}[!ht]
		\centering
		\includegraphics{img/2b.png}
		\caption{Polos e Zeros}	
	\end{figure}	
	\begin{figure}[!ht]
		\includegraphics{img/2c.png}
		\caption{Diagrama de Bode}	
	\end{figure}			
	\newpage
	\clearpage
	\subsection{Resposta ao degrau unitário}
		\begin{figure}[!ht]
			\centering
			\includegraphics[scale=0.71]{img/2f.png}
			\caption{Resposta ao degrau unitário}	
		\end{figure}		
	
	\subsection{Resposta a rampa unitária}
		\begin{figure}[!ht]
			\centering
			\includegraphics[scale=0.71]{img/2g.png}
			\caption{Resposta a rampa unitária}	
		\end{figure}				

			\subsection{Resposta a onda quadrada}
			\begin{figure}[!ht]
				\centering
				\includegraphics[scale=0.71]{img/2h.png}
				\caption{Resposta a onda quadrada com $\omega = \frac{1}{4}\pi$}	
			\end{figure}			
			\begin{figure}[!ht]
				\centering
				\includegraphics[scale=0.71]{img/2i.png}
				\caption{Resposta ao primeiro harmônico da série de Fourier de um onda quadrada com $\omega = \frac{1}{2}\pi$}	
			\end{figure}		
			\begin{figure}[!ht]
				\centering
				\includegraphics[scale=0.71]{img/2j.png}
				\caption{Resposta ao terceiro harmônico da série de Fourier de um onda quadrada com $\omega = \frac{1}{2}\pi$}	
			\end{figure}			
			\begin{figure}[!ht]
				\centering
				\includegraphics[scale=0.71]{img/2k.png}
				\caption{Resposta ao quinto harmônico da série de Fourier de um onda quadrada com $\omega = \frac{1}{2}\pi$}	
			\end{figure}		
			\begin{figure}[!ht]
				\centering
				\includegraphics[scale=0.71]{img/2l.png}
				\caption{Resposta ao sétimo harmônico da série de Fourier de um onda quadrada com $\omega = \frac{1}{2}\pi$}	
			\end{figure}		
			\clearpage
							
	\section{Quest\~{a}o 3}
		Resolva as questões para os sistemas descritos pelas seguintes funções de transferência:
		\[
		H(S)= \frac{1 + \alpha S}{S^{2} + 2S +  2}
		\] 	\\			
		\[
		H(S)= \frac{S + 10^{4}}{S^{2} + 2\beta S +  100}
		\] 	\\				
		
		\subsection{E.D.O dos sistemas}

			\subsubsection{Sistema 1}
				\[
				X(S)[1+ \alpha S] = Y(S)[S^{2} + 2S + 2] \Rightarrow
				\] 	\\			
				\[						
				\alpha \frac{\partial x(t)}{\partial t} + x(t) = \frac{\partial^{2}y(t)}{\partial^{2}t} + 2\frac{\partial^y(t)}{\partial t} + 2y(t)
				\] 	\\				
			\subsubsection{Sistema 2}
				\[
				X(S)[S + 10^{4}] = Y(S)[S^{2} + 2\beta S +  100] \Rightarrow
				\] 	\\			
				\[						
				\frac{\partial x(t)}{\partial t} + 10^{4}x(t) = \frac{\partial^{2}y(t)}{\partial^{2}t} + 2 \beta\frac{\partial y(t)}{\partial t} + 100y(t)
				\] 	\\				
				\newpage
		\subsection{Polos e zeros}	
			\subsubsection{Variando em $\alpha$}
			\begin{figure}[!ht]
				\centering
				\includegraphics[scale=0.5]{img/3b_alfa.png}
				\caption{Polos e Zeros variando em $\alpha$}	
			\end{figure}				
			\subsubsection{Variando em $\beta$}		
			\begin{figure}[!ht]
				\centering
				\includegraphics[scale=0.55]{img/3b_beta.png}
				\caption{Polos e Zeros variando em $\beta$}	
			\end{figure}					
		\subsection{Diagrama de Bode}		
			\subsubsection{Variando em $\alpha$}
			\begin{figure}[!ht]
				\centering
				\includegraphics[scale=0.42]{img/3c_alfa.png}
				\caption{Diagrama de Bode variando em $\alpha$}	
			\end{figure}				
			\subsubsection{Variando em $\beta$}	
			\begin{figure}[!ht]
				\centering
				\includegraphics[scale=0.46]{img/3c_beta.png}
				\caption{Diagrama de Bode variando em $\beta$}	
			\end{figure}			
		\subsection{Resposta ao Degrau Unitário}		
			\subsubsection{Variando em $\alpha$}
			\begin{figure}[!ht]
				\centering
				\includegraphics[scale=0.45]{img/3d_alfa.png}
				\caption{Resposta ao Degrau Unitário variando em $\alpha$}	
			\end{figure}				
			\subsubsection{Variando em $\beta$}
			\begin{figure}[!ht]
				\centering
				\includegraphics[scale=0.5]{img/3d_beta.png}
				\caption{Resposta ao Degrau Unitário variando em $\beta$}	
			\end{figure}	
    	\newpage				
		\subsection{Resposta a Rampa Unitária}		
		\subsubsection{Variando em $\alpha$}
		\begin{figure}[!ht]
			\centering
			\includegraphics[scale=0.5]{img/3e_alfa.png}
			\caption{Resposta a rampa Unitária variando em $\alpha$}	
		\end{figure}				
		\subsubsection{Variando em $\beta$}
		\begin{figure}[!ht]
			\centering
			\includegraphics[scale=0.45]{img/3e_beta.png}
			\caption{Resposta a rampa Unitária variando em $\beta$}	
		\end{figure}
		\subsection{Resposta a onda quadrada}					
			\subsubsection{Variando em $\alpha$}
			\begin{figure}[!ht]
				\centering
				\includegraphics[scale=0.4]{img/3f_alfa.png}
				\caption{Resposta a onda quadrada com $\omega = \frac{1}{8}\pi$}	
			\end{figure}			
			\begin{figure}[!ht]
				\centering
				\includegraphics[scale=0.5]{img/3g_alfa.png}
				\caption{Resposta ao primeiro harmônico da série de Fourier de um onda quadrada com $\omega = \frac{1}{4}\pi$}	
			\end{figure}		
			\begin{figure}[!ht]
				\centering
				\includegraphics[scale=0.5]{img/3h_alfa.png}
				\caption{Resposta ao terceiro harmônico da série de Fourier de um onda quadrada com $\omega = \frac{1}{4}\pi$}	
			\end{figure}			
			\begin{figure}[!ht]
				\centering
				\includegraphics[scale=0.5]{img/3i_alfa.png}
				\caption{Resposta ao quinto harmônico da série de Fourier de um onda quadrada com $\omega = \frac{1}{4}\pi$}	
			\end{figure}		
			\begin{figure}[!ht]
				\centering
				\includegraphics[scale=0.57]{img/3j_alfa.png}
				\caption{Resposta ao sétimo harmônico da série de Fourier de um onda quadrada com $\omega = \frac{1}{4}\pi$}	
			\end{figure}	
			\clearpage		
			\subsubsection{Variando em $\beta$}
			\begin{figure}[!ht]
				\centering
				\includegraphics[scale=0.46]{img/3f_beta.png}
				\caption{Resposta a onda quadrada com $\omega = \frac{1}{8}\pi$}	
			\end{figure}			
			\begin{figure}[!ht]
				\centering
				\includegraphics[scale=0.57]{img/3g_beta.png}
				\caption{Resposta ao primeiro harmônico da série de Fourier de um onda quadrada com $\omega = \frac{1}{4}\pi$}	
			\end{figure}		
			\begin{figure}[!ht]
				\centering
				\includegraphics[scale=0.52]{img/3h_beta.png}
				\caption{Resposta ao terceiro harmônico da série de Fourier de um onda quadrada com $\omega = \frac{1}{4}\pi$}	
			\end{figure}			
			\begin{figure}[!ht]
				\centering
				\includegraphics[scale=0.48]{img/3i_beta.png}
				\caption{Resposta ao quinto harmônico da série de Fourier de um onda quadrada com $\omega = \frac{1}{4}\pi$}	
			\end{figure}		
			\begin{figure}[!ht]
				\centering
				\includegraphics[scale=0.5]{img/3j_beta.png}
				\caption{Resposta ao sétimo harmônico da série de Fourier de um onda quadrada com $\omega = \frac{1}{4}\pi$}	
			\end{figure}					
	
	\subsection{Resposta a cossenoides}	
	\subsubsection{Variando frequências nos valores de $\alpha$}	
			\begin{figure}[!ht]
				\centering
				\includegraphics[scale=0.45]{img/3k_alfa1.png}
				\caption{Resposta para $\alpha$ = 0.001 em frequências variantes}	
			\end{figure}		
			\begin{figure}[!ht]
				\centering
				\includegraphics[scale=0.48]{img/3k_alfa2.png}
				\caption{Resposta para $\alpha$ = 0.01 em frequências variantes}	
			\end{figure}			
			\begin{figure}[!ht]
				\centering
				\includegraphics[scale=0.48]{img/3k_alfa3.png}
				\caption{Resposta para $\alpha$ = 0.1 em frequências variantes}	
			\end{figure}				
			\begin{figure}[!ht]
				\centering
				\includegraphics[scale=0.52]{img/3k_alfa4.png}
				\caption{Resposta para $\alpha$ = 1 em frequências variantes}	
			\end{figure}		
			\begin{figure}[!ht]
				\centering
				\includegraphics[scale=0.52]{img/3k_alfa5.png}
				\caption{Resposta para $\alpha$ = 10 em frequências variantes}	
			\end{figure}		
			\begin{figure}[!ht]
				\centering
				\includegraphics[scale=0.48]{img/3k_alfa6.png}
				\caption{Resposta para $\alpha$ = 100 em frequências variantes}	
			\end{figure}		
			\begin{figure}[!ht]
				\centering
				\includegraphics[scale=0.49]{img/3k_alfa7.png}
				\caption{Resposta para $\alpha$ = 1000 em frequências variantes}	
			\end{figure}				
			\clearpage							
	\subsubsection{Variando frequências nos valores de $\beta$}
			\begin{figure}[!ht]
				\centering
				\includegraphics[scale=0.45]{img/3k_beta1.png}
				\caption{Resposta para $\beta$ = 0.001 em frequências variantes}	
			\end{figure}		
			\begin{figure}[!ht]
				\centering
				\includegraphics[scale=0.52]{img/3k_beta2.png}
				\caption{Resposta para $\beta$ = 0.01 em frequências variantes}	
			\end{figure}			
			\begin{figure}[!ht]
				\centering
				\includegraphics[scale=0.45]{img/3k_beta3.png}
				\caption{Resposta para $\beta$ = 0.1 em frequências variantes}	
			\end{figure}				
			\begin{figure}[!ht]
				\centering
				\includegraphics[scale=0.45]{img/3k_beta4.png}
				\caption{Resposta para $\beta$ = 1 em frequências variantes}	
			\end{figure}		
			\begin{figure}[!ht]
				\centering
				\includegraphics[scale=0.52]{img/3k_beta5.png}
				\caption{Resposta para $\beta$ = 10 em frequências variantes}	
			\end{figure}	
	\section{Conclusão}
		
	\newpage
	\section{Referências}
	
	[1] https://en.wikipedia.org/wiki/Buffer\_amplifier; \\
	
	[2] https://en.wikipedia.org/wiki/Electronic\_filter;\\
	
	[3] https://en.wikipedia.org/wiki/Low-pass\_filter;\\
	
	[4] https://en.wikipedia.org/wiki/Band-pass\_filter;\\
	
	[5] https://en.wikipedia.org/wiki/Butterworth\_filter;\\
	
	[6] https://en.wikipedia.org/wiki/Sallen-Key\_topology;\\
	
	[7] https://en.wikipedia.org/wiki/Integrator;\\
	
	[8] https://en.wikibooks.org/wiki/Signals\_and\_Systems;\\	
	
	[9] http://www.lps.ufrj.br/\~natmourajr/EEL350/2016\_01/slides\_SL1.pdf; \\
	
	[10] B. P. Lathi, Linear Systems and Signals. Oxford, UK: Oxford University Press, 2nd ed., 2009. \\
	
	[11] A. V. Oppenheim, A. S. Willsky, and S. H. Nawab, Signals and Systems (2Nd Ed.). Upper Saddle River, NJ, USA: Prentice-Hall, Inc., 1996.
	
\end{document}